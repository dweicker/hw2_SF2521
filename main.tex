\documentclass[11pt,a4paper]{article}

\usepackage[T1]{fontenc}
\usepackage[utf8]{inputenc}
\usepackage[english]{babel}
\usepackage{lmodern}
%\usepackage{circuitikz}
\usepackage{color}
\usepackage{wrapfig}
\usepackage{placeins}
\usepackage{subfigure}
\usepackage{tabu}
\usepackage{fullpage}
\usepackage[squaren]{SIunits}
\usepackage{graphicx}
%\usepackage[pdftex]{graphicx}
\usepackage{epstopdf}
\usepackage{epsfig}
\usepackage{hyperref}
\usepackage{tikz}
\usepackage{tikz-qtree}
\usepackage{eurosym}
%\usepackage{chemist}
\usepackage{amsmath}
\usepackage{amssymb}
\usepackage{mathrsfs}
\usepackage{dsfont}% use $\mathds{1}$
\newcommand{\C}{\mathbb{C}}
\newcommand{\N}{\mathbb{N}}
\newcommand{\Z}{\mathbb{Z}}
\newcommand{\R}{\mathbb{R}}
\newcommand{\red}{\textcolor{red}}
\newcommand{\dis}{\displaystyle}
\newcommand{\dr}{\partial}
\newcommand{\txt}{\text}
\newcommand{\td}{\todo[inline]}
\newcommand{\ttt}{\texttt}
\newcommand{\itt}{\textit}

\usepackage{algorithm}
\usepackage{todonotes}
\usepackage[noend]{algpseudocode}

%\newtheorem{theoreme}			     {Théorème}	[chapter]
%\newtheorem{proposition}[theoreme]	 {Proposition}	
%\newtheorem{corollaire}	  [theoreme]	 {Corollaire}	
%\newtheorem{lemme}	      [theoreme]  {Lemme}		
%\newtheorem{definition}	         {Définition}[chapter]
%\theoremstyle{definition}
%\newtheorem{exemple}			     {Exemple}	[chapter]
%\newtheorem{contreexemple}[exemple]{Contre-exemple}
%\newtheorem{probleme}	             {Probl\`eme}[chapter]

\usepackage{listings}
\usepackage{textcomp}
\definecolor{listinggray}{gray}{0.9}
\definecolor{lbcolor}{rgb}{0.9,0.9,0.9}
\lstset{
	backgroundcolor=\color{lbcolor},
	tabsize=4,
	rulecolor=,
	language=matlab,
        basicstyle=\scriptsize,
        upquote=true,
        aboveskip={1.5\baselineskip},
        columns=fixed,
        showstringspaces=false,
        extendedchars=true,
        breaklines=true,
        prebreak = \raisebox{0ex}[0ex][0ex]{\ensuremath{\hookleftarrow}},
        frame=single,
        showtabs=false,
        showspaces=false,
        showstringspaces=false,
        identifierstyle=\ttfamily,
        keywordstyle=\color[rgb]{0,0,1},
        commentstyle=\color[rgb]{0.133,0.545,0.133},
        stringstyle=\color[rgb]{0.627,0.126,0.941},
}

\DeclareMathOperator{\e}{e}

\title{Titre}
\author{David Weicker}
\date{\today}

\begin{document}
\tabulinesep=1.2mm
\begin{center}
\hrule
\begin{tabular}{c}
\\[0.005cm]
\Large{It's the title bitchess!~!!}\\[0.3cm] %THIS IS THE TITLE
\textsc{Thomas} Garrett  \& \textsc{Weicker} David\\[0.2cm]
$\text{6}^{\text{th}}$ November 2015\\[0.2cm] %THIS IS THE DATE
\end{tabular}
\hrule
\end{center}

\section{Stability of Numerical Schemes}
We have the general scheme 
\begin{align*}
U^{n+1}  &= Q(t_n) U^n + \Delta t F^n \\
U^0 &= g
\end{align*}
where $U^n \in \mathbb{R}^d$. 

\subsection{Duhamel’s Principle}
We are given the following discrete Duhamel’s Principle:
\begin{align}
U^n = S_h(t_n,0)g+\Delta t \sum_{\nu = 0}^{n-1} S_h(t_n , t_{\nu + 1}) F^{\nu} , 
\end{align}
where $t_n = n \Delta t$, and 
\begin{align*}
S_h(t,t) &= I, \quad t \in \mathbb{R} \\
S_h(t_{n+1,t_{\mu}}) &= Q(t_n)S_h(t_n,t_{\mu}).
\end{align*}
\\
We begin by showing that (1) holds by induction.

Base Case: $n = 0$
\begin{align*}
U^0 &= S_h(0,0)g+\Delta t \sum_{\nu = 0}^{-1} S_h(0 , t_{\nu + 1}) F^{\nu} \\
& = 0
\end{align*}
which fits the general scheme. 
Now we assume that the (1) fits the general scheme at step $n$, and we want to show that this implies that it fits for step $n+1$.
\begin{align*}
U^{n+1} &= S_h(t_{n+1},0)g+\Delta t \sum_{\nu = 0}^{n} S_h(t_{+1}n , t_{\nu + 1}) F^{\nu} \\
&= Q(t_n)S_h(t_n,0)g + \Delta t Q(t_n) \sum_{\nu = 0}^{n-1} S_h(t_n , t_{\nu + 1}) F^{\nu} + S_h(t_{n+1} , t_{n + 1}) F^{n} \\ &= Q(t_n)(S_h(t_n,0)g + \Delta t \sum_{\nu = 0}^{n-1} S_h(t_n , t_{\nu + 1}) F^{\nu})+F^{n} \\ 
&= Q(t_n)U_n+F^{n} %\quad \text{by induction assumption} 
\end{align*}
which fits our general scheme.
\subsection{Bound in the $h$-norm}
We now wish to show that 
\begin{align*}
||S_h(t_{\nu+1},t_{\nu})||_h \leq Ke^{ah} \implies ||U^n||_h \leq K(e^{at_n} ||g||_h + \int_0^{t_n} e^{a(t_n-s)} ds \max_{0\leq \nu \leq n-1} ||F^{\nu}||_h
\end{align*}
Taking $||\cdot ||_h$ of both sides of (1), then by Cauchy-Schwartz inequality, we have
\begin{align*}
||U^n||_h &= ||S_h(t_n,0)g+\Delta t \sum_{\nu = 0}^{n-1} S_h(t_n , t_{\nu + 1}) F^{\nu} ||_h \\
&\leq ||S_h(t_n,0)||_h||g||_h+||\Delta t||_h \sum_{\nu = 0}^{n-1} ||S_h(t_n , t_{\nu + 1})||_h || F^{\nu} ||_h \\
&\leq Ke^{at_n}||g||_h+||\Delta t||_h \sum_{\nu = 0}^{n-1} ||S_h(t_n , t_{\nu + 1})||_h || F^{\nu} ||_h\\
&\leq Ke^{at_n}||g||_h+\Delta t \sum_{\nu = 0}^{n-1} Ke^{a(t_n - t_{\nu+1})} || F^{\nu} ||_h\\ 
\text{we notice that} \enspace & \Delta t \sum_{\nu = 0}^{n-1} Ke^{a(t_n - t_{\nu+1})} \enspace \text{is a right Remman sum of a strictly decreasing function, thus} \\
&\leq Ke^{at_n}||g||_h+ K\int_0^{t_n} e^{a(t_n - s)} ds || F^{\nu} ||_h\\ 
&\leq K(e^{at_n}||g||_h+ \int_0^{t_n} e^{a(t_n - s)} ds \max_{0\leq \nu \leq n-1} ||F^{\nu}||_h)
\end{align*}
IS a POSITIVE??
\subsection{a Value}
If $a = h^{-1/2}$, then we would have $||S_h(t_{\nu + 1}, t_{\nu})||_h \leq Ke^{\sqrt{h}}$. Plugging this into our second inequality, we obtain, 
\begin{align*}
||U^n||_h\leq K(e^{\sqrt{t_n}}||g||_h+ \int_0^{t_n} e^{\sqrt{t_n - s}} ds \max_{0\leq \nu \leq n-1} ||F^{\nu}||_h)
\end{align*}


\end{document}