We have the general scheme 
\begin{align*}
U^{n+1}  &= Q(t_n) U^n + \Delta t F^n \\
U^0 &= g
\end{align*}
where $U^n \in \mathbb{R}^d$. 

\subsection{Duhamel’s Principle}
We are given the following discrete Duhamel’s Principle:
\begin{align}
U^n = S_{\Delta t}(t_n,0)g+\Delta t \sum_{\nu = 0}^{n-1} S_{\Delta t}(t_n , t_{\nu + 1}) F^{\nu} , 
\end{align}
where $t_n = n \Delta t$, and 
\begin{align*}
S_{\Delta t}(t,t) &= I, \quad t \in \mathbb{R} \\
S_{\Delta t}(t_{n+1,t_{\mu}}) &= Q(t_n)S_{\Delta t}(t_n,t_{\mu}).
\end{align*}
\\
We begin by showing that (1) holds by induction.

Base Case: $n = 0$
\begin{align*}
U^0 & = S_{\Delta t}(0,0)g+\Delta t \sum_{\nu = 0}^{-1} S_{\Delta t}(0 , t_{\nu + 1}) F^{\nu} \\
&= S_{\Delta t}(0,0)g+\Delta t 0 \\
&= g
\end{align*}

This is indeed true by the scheme given.
 
Now we assume that the discrete Duhamel's Principle fits the general scheme at step $n$, and we want to show that this implies that it fits for step $n+1$.
\begin{align*}
U^{n+1} &= Q(t_n)U_n+F^{n} \\ 
&= Q(t_n)(S_{\Delta t}(t_n,0)g + \Delta t \sum_{\nu = 0}^{n-1} S_{\Delta t}(t_n , t_{\nu + 1}) F^{\nu})+F^{n} \quad \text{by induction assumption} \\ 
&= Q(t_n)S_{\Delta t}(t_n,0)g + \Delta t Q(t_n) \sum_{\nu = 0}^{n-1} S_{\Delta t}(t_n , t_{\nu + 1}) F^{\nu} + S_{\Delta t}(t_{n+1} , t_{n + 1}) F^{n} \\ 
&= S_{\Delta t}(t_{n+1},0)g+\Delta t \sum_{\nu = 0}^{n} S_{\Delta t}(t_{n+1} , t_{\nu + 1}) F^{\nu}
\end{align*}

So we have proven that Duhamel's principle holds for the given numerical scheme.

Duhamel's principle can give us an extra understanding on how an inhomogeneous term affects the solution. If $F^\nu = 0$ then the solution is given by :
$$U^n = S_{\Delta t}(t_n,0)g$$

So the initial conditions $g$ is affected by $S$ and carried from $0$ to $t_n$. If now, $F^\nu \neq 0$, we can see by Duhamel's principle that we can look at the inhomogeneous term $F^\nu$ as an initial condition at time $t_\nu$. This initial condition is affected by the same $S$ but now carried from $t_\nu$ to $t_n$. The solution is then given by the sum of all those "initial conditions" carried to $t_n$.


\subsection{Bound in the $\Delta t$-norm}
We now wish to show that 
\begin{align*} 
||S_{\Delta t}(t_{\nu+1},t_{\nu})||_{\Delta t} \leq Ke^{a \Delta t} \implies ||U^n||_{\Delta t} \leq K(e^{at_n} ||g||_{\Delta t} + \int_0^{t_n} e^{a(t_n-s)} ds \max_{0\leq \nu \leq n-1} ||F^{\nu}||_{\Delta t}
\end{align*}
Taking $||\cdot ||_{\Delta t}$ of both sides of (1), then by Cauchy-Schwartz inequality, we have
\begin{align*}
||U^n||_{ \Delta t} &= ||S_{\Delta t}(t_n,0)g+\Delta t \sum_{\nu = 0}^{n-1} S_{\Delta t}(t_n , t_{\nu + 1}) F^{\nu} ||_{\Delta t} \\
&\leq ||S_{\Delta t}(t_n,0)||_{\Delta t}||g||_{\Delta t}+\Delta t \sum_{\nu = 0}^{n-1} ||S_{\Delta t}(t_n , t_{\nu + 1})||_{\Delta t} || F^{\nu} ||_{\Delta t} \\
&\leq Ke^{at_n}||g||_{\Delta t}+\Delta t \sum_{\nu = 0}^{n-1} ||S_{\Delta t}(t_n , t_{\nu + 1})||_{\Delta t} || F^{\nu} ||_{\Delta t}\\
&\leq Ke^{at_n}||g||_{\Delta t}+\Delta t \sum_{\nu = 0}^{n-1} Ke^{\alpha(t_n - t_{\nu+1})} || F^{\nu} ||_{\Delta t}\\ 
&\text{We can assume that $\alpha$ is positive because otherwise increasing $\Delta t$ will decrease the bound.}\\
&\text{We notice that} \enspace  \Delta t \sum_{\nu = 0}^{n-1} Ke^{a(t_n - t_{\nu+1})} \enspace \text{is a right Remman sum of a strictly decreasing function, thus} \\
&\leq Ke^{at_n}||g||_{\Delta t}+ K\int_0^{t_n} e^{\alpha(t_n - s)} ds || F^{\nu} ||_{\Delta t}\\ 
&\leq K(e^{at_n}||g||_{\Delta t}+ \int_0^{t_n} e^{\alpha(t_n - s)} ds \max_{0\leq \nu \leq n-1} ||F^{\nu}||_{\Delta t})
\end{align*}
\subsection{$\alpha$ Value}
If $a = \Delta t^{-1/2}$, then we would have $||S_{\Delta t}(t_{\nu + 1}, t_{\nu})||_{\Delta t} \leq Ke^{\sqrt{\Delta t}}$. Plugging this into our second inequality, we obtain, 
\begin{align*}
||U^n||_{\Delta t}\leq K(e^{n\sqrt{t_n}}||g||_{\Delta t}+ \int_0^{t_n} e^{n\sqrt{t_n - s}} ds \max_{0\leq \nu \leq n-1} ||F^{\nu}||_{\Delta t})
\end{align*}

That means that for a given time $t_n$, if we increase the number of time steps $n$, the bound increases exponentially. Since it is always possible to increase $n$, that means that we have no bound at all. So we can no longer conclude that we have stability. 
