The shallow water equation can be written in quasilinear form as \begin{align*}
u_t + f'(u) u_x = 0
\end{align*}
where $ u = (h,hv)^T $ and 
\begin{align*}
f'(u) = \begin{pmatrix}
0 & 1 \\
-(\frac{u_2}{u_1})^2 + gu_1 & 2 (\frac{u_2}{u_1})
\end{pmatrix}
\end{align*}
Now, to make this system linear we must simply pick a constant state $(h_0,v_0)$ which is consistent with the boundary and initial conditions. We choose $(h_0,v_0)$ at $x=5$ and $t=0$. We can now compute $(h_0,v_0)$ using the initial condition. We get $(h_0,v_0) = (1.1,0)$. Thus we have 
\begin{align*}
f'(u) = \begin{pmatrix}
0 & 1 \\
1.1g & 0
\end{pmatrix} = \begin{pmatrix}
0 & 1 \\
10.571 & 0
\end{pmatrix}
\end{align*}
We can see easily $f'(u)$ has eigenvalues $\lambda_{1,2} = \pm \sqrt{10.571} = \pm 3.2513$ which are real. It also has a full set of eigenvectors i.e. $(1,3.2513)^T$ and $(-1,3.2513)$. This together confirm that the linear problem is hyperbolic. 